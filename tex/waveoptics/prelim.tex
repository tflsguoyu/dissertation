\section{Preliminaries}
\label{sec:waveoptics:prelim}

We now briefly revisit the basics on first principles of (classical) light transport theory based on Maxwell electromagnetism. Table \ref{tab:waveoptics:notation} summarize the symbols using along this chapter.

\begin{table}[h!]
	\centering
    \caption[Notation used in \S\ref{cpt:waveoptics}]{\label{tab:waveoptics:notation}
    	Notation used in \S\ref{cpt:waveoptics}.
    }
	\renewcommand{\arraystretch}{1.2}
    \begin{tabular}{ll}
        $\bfr\in\bbR^3$ & Position \\
        $\hatbfr\in\calSS$ & Direction to $\bfr$. \\
        $r\in\bbR$ & Distance. \\
        \hline
        $\varepsilon(\bfr)$ & Permittivity \\
        $\mu(\bfr)$ & Permeability \\
        $\omega$ & Wave angular frequency [1/s] \\
        $\lambda=2\pi\omega^{-1}$ & Wavelength [m] \\
        $k(\bfr)=\omega\sqrt{\varepsilon(\bfr)\mu(\bfr)}$ & Wavenumber at $\bfr$\\
        $m(\bfr)=k_2(\bfr)/k_1$ & Relative refractive index at $\bfr$ \\
        \hline
        $\bfH(\bfr)$ & Magnetic field at $\bfr$ \\
        $\bfE(\bfr)$   & Electric field at $\bfr$~\eqref{eq:vri}  \\
        $\bfEi(\bfr)$ & Incident electric field $\bfr$\\
        $\bfEs(\bfr)$ & Scattered electric field at $\bfr$~\eqref{eq:vri}\\ 
        $\bfE_0$ & Amplitude of a planar electric field \\
        $\bfEs_1(\hatbfr)$ & Far-field angular distribution of the scattered radiation  \\
        \hline
        $\dyad{G}$ & Free-space dyadic Green's function~\eqref{eq:greenfunc} \\
        $\dyad{T}$ & Dyad transition operator~\eqref{eq:dyadtransition}\\
        $g(\hatbfn,\bfr)$ & Planar field scalar propagator \\
        \hline
        \hline
        $V_i$ & Volume suspended by particle/cluster $i$ \\
        $\bfR_i\in\bbR^3$ & Representative position of particle/cluster $i$ \\
        $\hatbfR_{ij}\in\calSS$ & Direction from $\bfR_j$ to $\bfR_i$\\
        $R_{ij}\in\bbR$ & Distance from $\bfR_j$ to $\bfR_i$\\
        $\Ncls$ & Number of particles in a cluster \\
        \hline
        $\bfEs_i(\bfr)$ & Scattered field of $\bfr\in V_i$~\eqref{eq:foldylax}\\
        $\bfE_i(\bfr)$ & Exciting field in $\bfr\in V_i$ \\
        $\bfEe_{ij}(\bfr)$ & Partial exciting field in $\bfr\in V_i$ from particle $j$~\eqref{eq:excfield} \\
        $\dyad{A}_i^\text{near}(\hatbfni,\bfr)$ & Near-field scattering dyad of particle/cluster $i$~\eqref{eq:scatdyad_near}. \\
        $\dyad{A}_i(\hatbfnis)$ & Far-field scattering dyad of particle/cluster $i$~\eqref{eq:farscatdyad}. \\
        \hline
        \hline
        $\Ct(\hatbfni),\Cs(\hatbfni)$ & Extinction~\eqref{eq:crosstcluster} and scattering~\eqref{eq:crossscluster} cross-sections [m$^{2}$]\\
        $\fp(\hatbfnis)$ & Phase function~\eqref{eq:phasecluster}\\
        $\rho$ & Particles density [m$^{-3}$] \\
        $\sigmat(\hatbfni),\sigmas(\hatbfni)$ & Extinction~\eqref{eq:sigmatcluster} and scattering~\eqref{eq:sigmascluster} coefficients [m$^{-1}$]
    \end{tabular}
\end{table}

\subsection{Electromagnetic Scattering}
\label{ssec:prelim_maxwells}

The propagation of a time-harmonic monochromatic electromagnetic field with frequency $\omega$ is defined by the Maxwell curl equations as
\begin{equation}
    \begin{aligned}
        \Curl\bfE(\bfr) &= \Img\,\omega\,\mu(\bfr)\,\bfH(\bfr),\\
        \Curl\bfH(\bfr) &= \Img\,\omega\,\varepsilon(\bfr)\,\bfE(\bfr),
    \end{aligned}
    \label{eq:maxwell}
\end{equation}
where $\Curl\cdot$ is the curl operator; $\bfE(\bfr)$ and $\bfH(\bfr)$ indicate, respectively, the (vector-valued) electric and magnetic fields at $\bfr$; $\mu(\bfr)$ and $\varepsilon(\bfr)$ denote the (scalar-valued) magnetic permeability and electric permittivity at $\bfr$, respectively; and $\Img := \sqrt{-1}$ is the imaginary unit.

Assuming a non-magnetic medium satisfying $\mu(\bfr) = \mu_0$ with $\mu_0$ being the magnetic permeability of a vacuum, \Eq{eq:maxwell} reduces to the electric field wave equation
\begin{equation}
    \nabla^2\times\bfE(\bfr) - k^2\,\bfE(\bfr) = 0,
    \label{eq:efieldwave}
\end{equation}
where $k(\bfr) = \omega\sqrt{\varepsilon(\bfr)\mu_0}$ is the medium's wave number at $\bfr$. 

We now assume an infinite homogeneous isotropic medium with permittivity $\varepsilon_1$, filled with scatterers bounded by a finite disjoint region $V$, with potentially inhomogeneous permittivity $\varepsilon_2(\bfr)$. Under this assumption, we can solve \Eq{eq:efieldwave} by expressing it as the \emph{volume integral equation} (see \S 3.1 of Mishchenko's work \cite{mishchenko2006multiple} for a step-by-step derivation) as the sum of the incident field $\bfEi(\bfr)$ and the scattered field $\bfEs(\bfr)$ due to inhomogeneities in the medium in the form of scatterers:
\begin{align}
    \bfE(\bfr) & = \bfEi(\bfr) + \bfEs(\bfr) \\
    & =\bfEi(\bfr) + k_1^2\,\int_V [m^2(\bfr')-1] \,\dyad{G}(\bfr,\bfr') \,\bfE(\bfr') \intd \bfr',
    \label{eq:vri}
\end{align}
with $k_1$ the wave number at the hosting medium, $m(\bfr) = k_2(\bfr)/k_1$ the index of refraction of the interior regions $V$ with respect to the hosting medium, and $\dyad{G}(\bfr,\bfr')$ the free-space dyadic Green's function defined as:
\begin{equation}
    \dyad{G}(\bfr,\bfr') = \left(\dyad{I} + k_1^{-2}\,\nabla\otimes\nabla\right) \frac{\exp(\Img\,k_1 \,|\bfr-\bfr'|)}{4\pi\,|\bfr-\bfr'|},
    \label{eq:greenfunc}
\end{equation}
where $\dyad{I}$ is the identity dyad, and $. \otimes .$ denotes the dyadic product of two vectors. Intuitively, \Eq{eq:vri} models the scattering field as the superposition of the spherical wavelets resulting from a change of permitivitty (i.e. with $m(\bfr')\neq1$). Note also the recursive nature of \Eq{eq:vri}; we will deal with this recursivity in the following section, computing $\bfEs(\bfr)$ as a function of the incident field $\bfEi(\bfr)$. 

\subsection{Foldy-Lax Equations}
\label{ssec:foldy-lax}

We now consider a medium filled with $N$ finite discrete particles with volume $V_i$ and index of refraction $m_i(\bfr)$. Considering an incident E-field $\bfEi(\bfr)$, we can rewrite \Eq{eq:vri} as
\begin{equation}
    \bfE(\bfr) = \bfEi(\bfr) + \int_{\bbR^3} U(\bfr')\,\dyad{G}(\bfr,\bfr') \cdot \bfE(\bfr') \intd \bfr',
    \label{eq:EfieldParticles}
\end{equation}
where $\dyad{G}(\bfr,\bfr')$ is the dyadic Green's function~\eqref{eq:greenfunc}, and $U(\bfr)$ the potential function given by
\begin{equation}
    U(\bfr) = \sum_{i=1}^{N} U_i(\bfr) \quad \text{with} \quad U_i(\bfr) = \begin{cases} 
    0, & (\bfr \notin V_i)\\ 
    k_1^2[m_i^2(\bfr)-1]. & (\bfr \in V_i)
    \end{cases}
    \label{eq:potential}
\end{equation}
By combining \Eqs{eq:EfieldParticles}{eq:potential}, we can express the field at any position $\bfr\in\bbR^3$ following the so-called \emph{Foldy-Lax equation} \cite{foldy1945multiple,lax1951multiple} as
\begin{equation}
\bfE(\bfr) = \bfEi(\bfr) + \sum_{i=1}^N \overbrace{\int_{V_i} \dyad{G}(\bfr,\bfr')\cdot \int_{V_i} \dyad{T}_i(\bfr',\bfr'') \cdot  \bfE_i(\bfr'') \intd \bfr''\, \intd \bfr'}^{\eqdef \,\bfEs_i(\bfr)}    \label{eq:foldylax}
\end{equation}
with $\bfEs_i(\bfr)$ and $\bfE_i(\bfr)$ the scattered and partial field of particle $i$, and $\dyad{T}_j(\bfr,\bfr')$ $\dyad{T}_i(\bfr,\bfr')$ the dyad transition operator for particle $i$ defined as \cite{tsang1985theory} 
\begin{equation}
\label{eq:dyadtransition}
    \begin{split}
        \dyad{T}_i(\bfr,\bfr') =\;& U_i(\bfr) \,\delta(\bfr-\bfr')\,\dyad{I} + U_i(\bfr) \int_{V_i} \dyad{G}(\bfr,\bfr'') \cdot \dyad{T}(\bfr'',\bfr') \intd \bfr'',
    \end{split}
\end{equation}
with $\delta(x)$ the Dirac delta. 
The partial field at particle $i$ is defined as $\bfE_i(\bfr)=\bfEi(\bfr) + \sum_{j(\neq i)=1}^N \bfEe_{ij}(\bfr)$, where the partial exciting field $\bfEe_{ij}(\bfr)$ from particles $j$ to $i$ is 
\begin{equation}
\bfEe_{ij}(\bfr) = \int_{V_j} \dyad{G}(\bfr,\bfr')\int_{V_j} \dyad{T}_j(\bfr',\bfr'') \bfE_j(\bfr'') \intd \bfr''\,\intd \bfr',
\label{eq:excfield}
\end{equation}
with $\bfr\in V_i$. Note that the scattered and exciting fields for particle $j$ have essentially the same form. 
As shown by Mishchenko \cite{mishchenko2002vector}, the Foldy-Lax equation~\eqref{eq:foldylax} solves exactly the volume integral equation~\eqref{eq:vri} for multiple arbitrary particles in the medium, without any assumptions on their composition or packing rate, beyond the assumption of a homogeneous hosting medium.

\begin{figure}[h!]
	\centering
	\def\svgwidth{.5\textwidth}
	\input{img/waveoptics/scheme/diagram_fig1} \\ [8pt]  
	\textbf{(a)} \\
	\def\svgwidth{.5\textwidth}
	\input{img/waveoptics/scheme/diagram_fig2} \\ [-5pt]
	\textbf{(b)} \\ [5pt]
	\def\svgwidth{.5\textwidth}
	\input{img/waveoptics/scheme/diagram_fig3} \\ [-5pt]
	\textbf{(c)}
	
	\caption[Schematical representation of the particles scattering geometry]{\label{fig:waveoptics:diagram}
	  	\textbf{Schematical representation of the particles scattering geometry.} \textbf{(a)} Previous methods, including Lorenz-Mie theory, assume independent scattering of particles, assuming that the distance $\tPx_{ij}$ between two particles $i$ and $j$ is very large (i.e. $\tPx_{ij}\rightarrow\infty$), neglecting the potential interactions between particles. \textbf{(b)} In our work  we differentiate between near field scattering of particles within a small region in space (cluster $\Cls$ centered at $\Px_\Cls$), and particles $k$ on the far-field region of the cluster (distance $\tPx_{\Cls k}\rightarrow\infty$). \textbf{(c)} For large values of $\tPx_{\Cls k}$, the direction between particle $k$ and any particle $j\in\Cls$ is $dPx_{ik}\approx\dPx_{\Cls k}$: Therefore, we can assume an planar exciting field $\ExcEField(\px)_{\Cls k}$ on the whole cluster $\Cls$ from particle $k$, with direction $\dPx_{\Cls k}$. 
	}
\end{figure}

\paragraph{Far-field Foldy-Lax Equations}
\Eq{eq:excfield} defines the exact exciting field resulting from the scattering by particle~$j$ on particle~$i$.
However, if the distance $R_{ij} \defeq \| \bfR_i - \bfR_j \|$ between particles (with $\bfR_i$ denoting the center of particle $i$) is large, we can approximate the propagation distance between any point $\bfr \in V_i$ and $\bfr' \in V_j$ as
\begin{equation}
    \| \bfr - \bfr' \| \approx R_{ij} + (\hatbfR_{ij} \cdot {\Delta}\bfr) -  (\hatbfR_{ij} \cdot {\Delta}\bfr'),
\end{equation}
with $\hatbfR_{ij} \defeq (\bfR_i - \bfR_j)/R_{ij}$, ${\Delta}\bfr \defeq \bfr - \bfR_i$ and ${\Delta}\bfr' \defeq \bfr' - \bfR_j$ (see Figure \ref{fig:waveoptics:diagram}, left).
With this approximation, we can now express $\bfEe_{ij}(\bfr)$ for a point $\bfr \in V_i$ using its \emph{far-field} approximation, as:%
\footnote{We note that, accordingly to Mishchenko \cite{mishchenko2006multiple}, the product would require to multiply the integrand by the dyad $(\dyad{I} - \hatbfR_{ij}\otimes\hatbfR_{ij})$ to ensure a transverse planar field; we remove it for clarity.}
\begin{equation}
    \begin{split}
        & \bfEe_{ij}(\bfr)\\
        \approx\;& \frac{\Exp^{\Img k_1 (R_{ij}+\hatbfR_{ij}\cdot{\Delta}\bfr)}}{4\pi R_{ij}} \int_{V_j} g(\hatbfR_{ij},{\Delta}\bfr') \int_{V_j} \dyad{T}_j(\bfr',\bfr'')\cdot \bfE_j(\bfr'') \intd \bfr''\,\intd \bfr' \\
        =\;& \frac{\exp(\Img k_1 \,R_{ij})}{R_{ij}} 
        g(\hatbfR_{ij}, \Delta \bfr) \,\bfEe_{1ij}(\hatbfR_{ij}),
    \end{split}
    \label{eq:excfieldfar}
    \raisetag{17pt}
\end{equation}
where: $\bfr \in V_i$ is a point in particle $i$; $g(\hatbfn, \Delta \bfr)=\exp(\Img k_1 \,\hatbfR_{ij}\cdot{\Delta}\bfr)$; and $\bfEe_{1ij}$ is the far-field exciting field from particle $j$ to particle $i$ that is solely characterized by the propagation direction $\hatbfR_{ij}$. In order for \Eq{eq:excfieldfar} to be valid, the distance $R_{ij}$ needs to hold the far-field criteria, which relates the $R_{ij}$ with the radius of the particle $a_j$ following the inequality~\cite{mishchenko2006multiple}:
\begin{equation}
    k_1 R_{ij} \gg \max\left(1, \frac{k_1^2a_j^2}{2}\right).
    \label{eq:farfield}
\end{equation}
This far-field assumption is both the basis for the Lorenz-Mie theory \cite{hulst1981light} (to model electromagnetic scattering from small spherical particles) and, as shown by Mishchenko \cite{mishchenko2002vector}, at the core of the radiative transfer theory.

In the following, we relax the assumption of near field scattering and compute the Foldy-Lax equations for clusters of particles for both the near- and far-field regions. Then, we use them to compute the scattering matrix to be used in the RTE to efficiently approximate light transport between clusters of particles. 

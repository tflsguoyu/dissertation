\section{Scattering from Clusters of Particles}
\label{sec:waveoptics:ours_theory}

In this section, we present our main theoretical result: the far-field approximated scattering dyad relating a field incoming at a particle, which will be shown in \Eq{eq:farscatdyad}.
This dyad can then be used to compute a medium's bulk scattering parameters, which we will discuss in \S\ref{ssec:ours_RTT}.

The two forms of computing the exciting field from particle $j$ to $i$ [\Eqs{eq:excfield}{eq:excfieldfar}] suggest that we can consider two subsets of particles $j$ depending on their distance with respect to the point of interest $\bfr$: One set of $\Nnear$ particles in the near field and another set of $\Nfar$ particles in the far field. With that, we can now calculate the exciting field in particle $i$ as:
\begin{equation}
    \bfE_i(\bfr)= \bfEi(\bfr) + \sum_{j(\neq i)=1}^\Nnear \bfEe_{ij}(\bfr) + \sum_{k=1}^{\Nfar} \bfEe_{ik}(\bfr).
    \label{eq:foldylaxtwo}
\end{equation}

In what follows, we derive the far-field Foldy-Lax equations for groups of particles where a cluster of these particles are in their respective near-field region, while the other elements in the system are in the far field. For the simplicity of our derivations, we consider a single far-field incident field in the cluster, and assume that the far-field particles $k$ do not have neighbor particles in their respective near field region.

More formally, we now consider a cluster $\Cls$ of $N_\Cls$ particles, where all particles $i\in\Cls$ are in their respective near-field region, and that the particles of the cluster have a bounding sphere centered at $\bfR_\Cls$ with radius $a_\Cls$ (see Figure \ref{fig:waveoptics:diagram}, middle). 

Since both the incident field $\bfEi(\bfr)$ and the exciting field $\bfEe_{\Cls k}(\bfr)$ from particle $k$ are in the far-field region, we can assume both fields to be planar waves defined as
\begin{align}
    \label{eq:farincfieldcluster}
    \bfEi(\bfr) &= \bfEi_0 \,\exp(\Img k_1 \hatbfn \cdot \Delta \bfr) = \bfEi_0\,g(\hatbfn, \Delta \bfr) , \\
    \bfEe_{\Cls k}(\bfr) &= \bfEe_{0\Cls k}\,\exp(\Img k_1 \hatbfR_{\Cls k} \cdot \Delta \bfr) =  \bfEe_{0\Cls k}\,g(\hatbfR_{\Cls k}, \Delta \bfr), 
    \label{eq:farexcfieldcluster} 
\end{align}
with $\bfEi_0$ the amplitude of the planar incident field, $\hatbfn$ its direction, and $\Delta \bfr=\bfr-\bfR_\Cls$. Equivalently, $\bfEe_{0\Cls k}=\frac{\exp(\Img k_1 \,R_{\Cls k})}{R_{\Cls k}}\,\bfEe_{1\Cls k}(\hatbfR_{\Cls k})$  is the amplitude of the exciting field at $\Cls$ from particle $k$, and $\hatbfR_{\Cls k}$ its direction. 

Now, let us slightly abuse the dot product notation, remove the dependency on the spatial dependency on each term, and use $(\varphi_1 \cdot \varphi_2) = \int \varphi_1(x)\,\varphi_2(x) \intd x$ for scalar-valued functions $\varphi_1$ and $\varphi_2$. From the far-field assumptions, plugging \Eq{eq:foldylaxtwo} into the definition of the scattered field from particle $i\in\Cls$ in Equation~\eqref{eq:foldylax} (with $\Nnear=\Ncls$) yields
\begin{equation}
    \label{eq:scafieldcluster1}
    \begin{split}
        \bfEs_i(\bfr) &= \dyad{G} \cdot \dyad{T}_i\cdot \bfE_i\\
        & = \dyad{G} \cdot \dyad{T}_i \cdot \left[\bfEi + \sum_{k=1}^{\Nfar} \bfEe_{\Cls k}+ \sum_{j(\neq i)=1}^{\Ncls} \bfEe_{ij} \right].
    \end{split}        
\end{equation}
By recursively expanding $\bfEe_{ij}$ and some algebraic operations (see the supplemental for the full derivation), this results into 
\begin{align}
    \label{eq:scafieldcluster5}
    \bfEs_i(\bfr) &= \bfE_0 \, \dyad{G} \cdot \dyad{T}_i \cdot \Bigg[ g(\hatbfn) + \sum_{j(\neq i)=1}^{\Ncls} \left[...\right]_j^{g(\hatbfn)} \Bigg] \\
    & + \sum_{k=1}^{\Nfar} \bfEe_{0\Cls j}\,\left[ \dyad{G} \cdot \dyad{T}_i \cdot \Bigg[ g(\hatbfR_{\Cls k}) + \sum_{j(\neq i)=1}^{\Ncls} \left[...\right]_j^{g(\hatbfR_{\Cls k})}\Bigg]\right]. \nonumber 
\end{align}
where the "$[...]_l^\varphi$" term represents the recursivity as
\begin{equation}
    [...]_j^\varphi= \dyad{G} \cdot \dyad{T}_j \cdot \left[\varphi + \sum_{l(\neq j)=1}^{\Ncls} \left[...\right]_l^\varphi\right] \,.
\end{equation}
Note that each element in the sum in \Eq{eq:scafieldcluster5} above is the result of the amplitude of the far-field incident or exciting fields, and a series that encode all the near-field scattering in the cluster $\Cls$. We can thus define the scattering dyad $\dyad{A}_i^\text{near}(\hatbfni,\bfr)$ relating a unit-amplitude planar incident field at particle $i$ from direction $\hatbfni$ with the scattered field at point $\bfr$ as
\begin{equation}
    \dyad{A}_i^\text{near}(\hatbfni,\bfr) = \dyad{G} \cdot \dyad{T}_i\cdot \Bigg[ g(\hatbfni) + \sum_{j(\neq i)=1}^{\Ncls} \left[...\right]_j^{g(\hatbfni)} \Bigg].
    \label{eq:scatdyad_near}
\end{equation}
By considering constant $\bfEi_0$ and $\bfEe_{0\Cls k}$ for the whole cluster $\Cls$, we can compute the cluster's scattering dyad as:
\begin{equation}
    \dyad{A}_\Cls^\text{near}(\hatbfni,\bfr) = \sum_{i=1}^{N_\Cls} \dyad{A}_i(\hatbfni,\bfr),
    \label{eq:scatdyadcluster_near}
\end{equation}
which defines the scattered field for a unit-amplitude incoming planar field in a scene consisting of the particles forming cluster $\Cls$.
In practice, the scattering dyad $\dyad{A}_\Cls^\text{near}(\hatbfni,\bfr)$ can be computed numerically using standard methods from computational electromagnetics \cite{mishchenko2014electromagnetic}.


\paragraph{Far-field approximation}
\Eq{eq:scatdyad_near} represents the general form of the scattering dyad for particle $i$, which results into a five-dimensional function. Assuming that $\bfr$ is in the far-field region of a particle $i\in\Cls$, by using the far-field approximation of the scattered or exciting field~\eqref{eq:excfieldfar} (we refer to the supplemental document for the derivation), we get the scattered field by particle $i$ as
\begin{align}
    \bfEs_i(\bfr) \approx \frac{\Exp^{\Img k_1 R_i}}{R_i} \Big(\bfE_0 \,  \dyad{A}_i(\hatbfn,\hatbfR_i) 
    + \sum_{k=1}^{\Nfar} \bfEe_{0\Cls k} \, \dyad{A}_i(\hatbfR_{\Cls k},\hatbfR_i) \Big),
    \label{eq:farscatfield}
\end{align}
with $R_i=|\bfr-\bfR_i|$ and $\hatbfR_i=\frac{\bfr-\bfR_i}{R_i}$, and
\begin{equation}
    \label{eq:farscatdyad}
    \boxed{%
        \dyad{A}_i(\hatbfnis) = \frac{g(\hatbfns)\cdot \dyad{T}_i}{4\pi} \cdot\Bigg[ g(\hatbfni) + \sum_{j(\neq i)=1}^{\Ncls} \left[...\right]_j^{g(\hatbfni)} \Bigg].
    }
\end{equation}
Finally, since $\hatbfR_i\approx\hatbfR_\Cls$ for all particles $i\in\Cls$, we can approximate the far-field scattered field of cluster $\Cls$ as
\begin{equation}
    \bfEs_\Cls(\bfr) = \frac{\Exp^{\Img k_1 R_\Cls}}{R_\Cls}\Big( \bfE_0 \,  \dyad{A}_\Cls(\hatbfn,\hatbfR_\Cls) + \sum_{k=1}^{\Nfar} \bfEe_{0\Cls k} \, \dyad{A}_\Cls(\hatbfR_{\Cls k},\hatbfR_\Cls) \Big),
    \label{eq:farscatfieldcluster}
\end{equation}
where
\begin{equation}
   \dyad{A}_\Cls(\hatbfnis) = \sum_{i=1}^{N_\Cls}\dyad{A}_i(\hatbfnis),
   \label{eq:farscatdyadC}
\end{equation}
is the far-field scattering dyad of cluster $\Cls$.

Thus, by grouping the individual particles into $\Ncls$ near-field clusters, and assuming that all clusters and observation point $\bfr$ lay in their respective far field, we can approximate the Foldy-Lax equation~\eqref{eq:foldylax} as
\begin{equation}
    \bfE(\bfr) = \bfEi(\bfr) + \sum_{\Cls_j=1}^{\Ncls} \bfEs_{\Cls_j}(\bfr),
    \label{eq:foldylaxcluster}
\end{equation}
with $\bfEs_{\Cls_j}(\bfr)$ the scattered field at cluster $\Cls_j$. 


\subsection{Relationship with the Radiative Transfer Theory}
\label{ssec:ours_RTT}

The scattering dyad $\dyad{A}_\Cls(\hatbfnis)$ given by \Eq{eq:farscatdyadC} models how a particle cluster $\Cls$ scatters a planar unit-amplitude incident field in the far field region. However, for rendering we are generally interested on the average field intensity (i.e. radiance). 

As shown by Mishchenko \cite{mishchenko2002vector}, the radiative transfer equation (RTE) directly derives from the far-field Foldy-Lax equations under three additional assumptions: (i)~The amount of coherent backscattering is negligible; (ii)~The particles are randomly distributed according to some distribution $p(R_i,\xi_i)$, with $R_i$ and $\xi$ denoting, respectively, the position and properties of a particle $i$; and (iii) We are interested on the average field $\EV{\bfE(\bfr)}$. 

Following these assumptions, and after a lengthy derivation, Mishchenko demonstrates that the bulk scattering properties can be obtained from the far-field Foldy-Lax form, and in particular from the scattering dyad $\dyad{A}(\hatbfnis)$. Let us first assume that the distribution of particle properties $\xi$ are independent of the particles position, and compute the average scattering dyad $\EV{\dyad{A}(\hatbfnis)} = \int_\Omega \dyad{A}_i(\hatbfnis) p(\xi_i) \intd \xi_i$. 

Then, note that the Foldy-Lax equation for clusters of particles~\eqref{eq:foldylaxcluster}, we derived above has the same form as the original Foldy-Lax equation~\eqref{eq:foldylax}. Thus, by the same derivation followed by Mishchenko we get to an equivalent RTE based on the scattering dyad of clusters. 

\paragraph{Computing the scattering parameters}
By taking the vectors of the parallel and perpendicular polarization $\hatbmthetai$ and $\hatbmphii$ of the incident field as shown in Figure \ref{fig:waveoptics:diagram} (right), and equivalently for the scattered field $\hatbmthetas$ and $\hatbmphis$, we can compute the polarized scattering components $\bmStheta$ and $\bmSphi$ from the cluster's scattering dyad $\dyad{A}_\Cls(\hatbfnis)$ as
\begin{align}
  \bmStheta(\hatbfnis) &= \hatbmthetas \cdot \EV{\dyad{A}_\Cls(\hatbfnis)} \cdot \hatbmthetai, \nonumber \\
  \bmSphi(\hatbfnis) &= \hatbmphis \cdot \EV{\dyad{A}_\Cls(\hatbfnis)} \cdot \hatbmphii.
\end{align}
Then, based on the two scattering components $\bmStheta$ and $\bmSphi$, we can obtain the optical parameters of the medium as
\begin{align}
    \label{eq:crosstcluster}
    \Ct(\hatbfni) &= 4\pi \Re\left[\frac{\bmS(\hatbfni,\hatbfni)}{k_i^2}\right], \\
    \label{eq:crossscluster}
    \Cs(\hatbfni) &=\int_\calSS \frac{|\bmStheta(\hatbfnis)|^2+|\bmSphi(\hatbfnis)|^2}{2k_1^2} \intd \hatbfns, \\
    \label{eq:phasecluster}
    \fp(\hatbfnis) &= \frac{|\bmStheta(\hatbfnis)|^2+|\bmSphi(\hatbfnis)|^2}{2k_1^2\Cs},
\end{align}
with $\bmS(\hatbfni,\hatbfni)=\bmSphi(\hatbfni,\hatbfni)=\bmStheta(\hatbfni,\hatbfni)$, and $\Re[x]$ returning the real part of a complex number $x$. Lastly, assuming a uniform distribution of clusters, we can compute the extinction and scattering coefficients as
\begin{align}
    \label{eq:sigmatcluster}
    \sigmat(\hatbfni) &= \Ct(\hatbfni) \frac{\rho}{\EV{\Ncls}}, \\
    \label{eq:sigmascluster}
    \sigmas(\hatbfni) &= \Cs(\hatbfni) \frac{\rho}{\EV{\Ncls}},
\end{align}
with $\rho$ the number of particles per differential volume, and $\EV{\Ncls}$ the average number of particles per cluster. Note that the optical properties defined in Equations~\eqref{eq:crosstcluster}--\eqref{eq:sigmascluster} are directionally dependent, so they are general and can represent both isotropic and anisotropic media. 


\subsection{Relationship with Independent Scattering}
\label{ssec:ours_indep_scat}

Most previous works rendering light transport in media \cite{novak2018monte} build on the assumption of independent scattering---that is, particles are in their respective far-field region.
It is easy to verify that this is a special case of \Eq{eq:foldylaxtwo} with $\Ncls=1$, causing 
the scattering dyad $\dyad{A}_\Cls$ of \Eq{eq:farscatdyadC} to reduce to
\begin{equation}
    \label{eq:farscatmie}
    \dyad{A}_\Cls(\hatbfnis) = \dyad{A}_i(\hatbfnis) = \frac{g(\hatbfns)\cdot \dyad{T}_i \cdot g(\hatbfni)}{4\pi},
\end{equation}
which encodes the scattered field in the far-field region of a particle when excited by an incident unit-amplitude planar field. 

The Lorenz-Mie theory \cite{hulst1981light} provides closed-form expressions for $\dyad{A}_i(\hatbfnis)$ for spheres and cylinders, while numerical solutions of $\dyad{A}_i(\hatbfnis)$ have been proposed for scatterers of arbitrary shapes via, for example, the T-matrix method \cite{waterman1965matrix}, or more recently based on the BEM for cylindrical fibers~\cite{xia2020wave}. Our work is therefore a generalization of these works to particles in the near field. 

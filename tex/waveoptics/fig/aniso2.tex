\begin{figure}[!ht]
    \centering
    \setlength{\resLen}{1.4in}
    \addtolength{\tabcolsep}{-3.5pt}
    \begin{tabular}{ccc}
        \begin{overpic}[width=\resLen]{waveoptics/lucy/color_aniso_x_700nm.jpg}
            \put(2,2){\color{white} \textbf{700nm-x}}
        \end{overpic}
        &
        \begin{overpic}[width=\resLen]{waveoptics/lucy/color_aniso_y_700nm.jpg}
            \put(2,2){\color{white} \textbf{700nm-y}}
        \end{overpic}
        &
        \begin{overpic}[width=\resLen]{waveoptics/lucy/color_aniso_z_700nm.jpg}
            \put(2,2){\color{white} \textbf{700nm-z}}
        \end{overpic} \\
        \begin{overpic}[width=\resLen]{waveoptics/lucy/color_aniso_x.jpg}
			\put(2,2){\color{white} \textbf{Multi.-x}}
		\end{overpic}
		&
		\begin{overpic}[width=\resLen]{waveoptics/lucy/color_aniso_y.jpg}
			\put(2,2){\color{white} \textbf{Multi.-y}}
		\end{overpic}
		&
		\begin{overpic}[width=\resLen]{waveoptics/lucy/color_aniso_z.jpg}
			\put(2,2){\color{white} \textbf{Multi.-z}}
		\end{overpic}    
    \end{tabular}
    \caption[Renderings of homogeneous Lucy models]{\label{fig:waveoptics:aniso2}
        \textbf{Renderings of homogeneous Lucy models} with the same anisotropic medium as in Figure \ref{fig:waveoptics:aniso1}.
        With the medium's orientation -- which determines the axis of the disk -- aligned with the $x$-, $y$-, and $z$-axis, respectively, the Lucy model exhibit distinct appearances.
    }
\end{figure}

\begin{figure}[h]
    \centering
    \setlength{\resLen}{1.5in}
    \addtolength{\tabcolsep}{0pt}
    \begin{tabular}{ccc}
        \begin{overpic}[width=\resLen]{waveoptics/lucy/aniso_x.jpg}
            \put(2,2){\color{white} \textbf{x}}
        \end{overpic}
        &
        \begin{overpic}[width=\resLen]{waveoptics/lucy/aniso_y.jpg}
            \put(2,2){\color{white} \textbf{y}}
        \end{overpic}
        &
        \begin{overpic}[width=\resLen]{waveoptics/lucy/aniso_z.jpg}
            \put(2,2){\color{white} \textbf{z}}
        \end{overpic}
    \end{tabular}
    \caption[Renderings of homogeneous Lucy models]{\label{fig:waveoptics:aniso2}
        \textbf{Renderings of homogeneous Lucy models} with the same anisotropic medium as in Figure \ref{fig:waveoptics:aniso1}.
        With the medium's orientation -- which determines the axis of the disk -- aligned with the $x$-, $y$-, and $z$-axis, respectively, the Lucy model exhibit distinct appearances.
    }
\end{figure}

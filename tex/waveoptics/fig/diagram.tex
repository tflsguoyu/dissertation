\begin{figure}[h!]
	\centering
	\def\svgwidth{.5\textwidth}
	\input{img/waveoptics/scheme/diagram_fig1} \\ [8pt]  
	\textbf{(a)} \\
	\def\svgwidth{.5\textwidth}
	\input{img/waveoptics/scheme/diagram_fig2} \\ [-5pt]
	\textbf{(b)} \\ [5pt]
	\def\svgwidth{.5\textwidth}
	\input{img/waveoptics/scheme/diagram_fig3} \\ [-5pt]
	\textbf{(c)}
	
	\caption[Schematical representation of the particles scattering geometry]{\label{fig:waveoptics:diagram}
	  	\textbf{Schematical representation of the particles scattering geometry.} \textbf{(a)} Previous methods, including Lorenz-Mie theory, assume independent scattering of particles, assuming that the distance $R_{ij}$ between two particles $i$ and $j$ is very large (i.e. $R_{ij}\rightarrow\infty$), neglecting the potential interactions between particles. \textbf{(b)} In our work  we differentiate between near field scattering of particles within a small region in space (cluster $\Cls$ centered at $\bfR_\Cls$), and particles $k$ on the far-field region of the cluster (distance $R_{\Cls k}\rightarrow\infty$). \textbf{(c)} For large values of $R_{\Cls k}$, the direction between particle $k$ and any particle $j\in\Cls$ is $dPx_{ik}\approx\hatbfR_{\Cls k}$: Therefore, we can assume an planar exciting field $\bfEe(\bfr)_{\Cls k}$ on the whole cluster $\Cls$ from particle $k$, with direction $\hatbfR_{\Cls k}$. 
	}
\end{figure}
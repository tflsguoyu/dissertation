\chapter{Introduction}
\label{cpt:introduction}

In this dissertation, we first address a more general but efficient way to handle complex surface reflectance and volumetric scattering, 


Next, we present an optimization based method for SVBRDF reconstruction and then extend it to bayesian inference.

To summarize, we develop a smart technique to render layered material, a framework to compute scatterings in participating media based on wave optics, and given a number of images, how to estimate the material properties. 
These techniques were presented at multiple conferences~\cite{guo2018position, guo2020materialgan, guo2020bayesian}. Our specific contributions include:

\paragraph{Position-free Monte Carlo simulation for arbitrary layered BSDFs.}
Real-world materials are often layered: metallic paints, biological tissues, and many more. Variation in the interface and volumetric scattering properties of the layers leads to a rich diversity of material appearances from anisotropic highlights to complex textures and relief patterns. However, simulating light-layer interactions is a challenging problem. Past analytical or numerical solutions either introduce several approximations and limitations, or rely on expensive operations on discretized BSDFs, preventing the ability to freely vary the layer properties spatially. We introduce a new unbiased layered BSDF model based on Monte Carlo simulation, whose only assumption is the layer assumption itself. Our novel position-free path formulation is fundamentally more powerful at constructing light transport paths than generic light transport algorithms applied to the special case of flat layers, since it is based on a product of solid angle instead of area measures, so does not contain the high-variance geometry terms needed in the standard formulation. We introduce two techniques for sampling the position-free path integral, a forward path tracer with next-event estimation and a full bidirectional estimator. We show a number of examples, featuring multiple layers with surface and volumetric scattering, surface and phase function anisotropy, and spatial variation in all parameters.
\chapter{Introduction}
\label{cpt:introduction}

In this dissertation, we first address a more general but efficient way to handle complex surface reflectance and volumetric scattering, 


Next, we present an optimization based method for SVBRDF reconstruction and then extend it to bayesian inference.

To summarize, we develop a smart technique to render layered material, a framework to compute scatterings in participating media based on wave optics, and given a number of images, how to estimate the material properties. 
These techniques were presented at multiple conferences \cite{guo2018position, guo2020materialgan, guo2020bayesian}. Our specific contributions include:

\paragraph{Position-free Monte Carlo simulation for arbitrary layered BSDFs.}
Real-world materials are often layered: metallic paints, biological tissues, and many more. Variation in the interface and volumetric scattering properties of the layers leads to a rich diversity of material appearances from anisotropic highlights to complex textures and relief patterns. However, simulating light-layer interactions is a challenging problem. Past analytical or numerical solutions either introduce several approximations and limitations, or rely on expensive operations on discretized BSDFs, preventing the ability to freely vary the layer properties spatially. 
In Chapter \ref{cpt:layeredbsdf}, we introduce a new unbiased layered BSDF model based on Monte Carlo simulation, whose only assumption is the layer assumption itself. Our novel position-free path formulation is fundamentally more powerful at constructing light transport paths than generic light transport algorithms applied to the special case of flat layers, since it is based on a product of solid angle instead of area measures, so does not contain the high-variance geometry terms needed in the standard formulation. We introduce two techniques for sampling the position-free path integral, a forward path tracer with next-event estimation and a full bidirectional estimator. We show a number of examples, featuring multiple layers with surface and volumetric scattering, surface and phase function anisotropy, and spatial variation in all parameters.

\paragraph{Beyond Mie theory: systematic computation of bulk scattering parameters based on microphysical wave optics.}
Light scattering in participating media and translucent materials is typically modeled using the radiative transfer theory. Under the assumption of independent scattering between particles, it utilizes several bulk scattering parameters to statistically characterize light-matter interactions at the macroscale. To calculate these parameters based on microscale material properties, the Lorenz-Mie theory has been considered the gold standard.
In Chapter \ref{cpt:waveoptics}, we present a generalized framework capable of systematically and rigorously computing bulk scattering parameters beyond the far-field assumption of Lorenz-Mie theory. Our technique accounts for microscale wave-optics effects such as diffraction and interference as well as interactions between nearby particles. Our framework is general, can be plugged in any renderer supporting Lorenz-Mie scattering, and allows arbitrary packing rates and particles correlation; we demonstrate this generality by computing bulk scattering parameters for a wide range of materials, including anisotropic and correlated media.

\paragraph{MaterialGAN: reflectance capture using a generative SVBRDF model.}
We address the problem of reconstructing spatially-varying BRDFs from a small set of image measurements. This is a fundamentally under-constrained problem, and previous work has relied on using various regularization priors or on capturing many images to produce plausible results.
In Chapter \ref{cpt:svbrdf}, we present \emph{MaterialGAN}, a deep generative convolutional network based on StyleGAN2, trained to synthesize realistic SVBRDF parameter maps. We show that MaterialGAN can be used as a powerful material prior in an inverse rendering framework: we optimize in its latent representation to generate material maps that match the appearance of the captured images when rendered. We demonstrate this framework on the task of reconstructing SVBRDFs from images captured under flash illumination using a hand-held mobile phone. Our method succeeds in producing plausible material maps that accurately reproduce the target images, and outperforms previous state-of-the-art material capture methods in evaluations on both synthetic and real data. Furthermore, our GAN-based latent space allows for high-level semantic material editing operations such as generating material variations and material morphing.

\paragraph{A Bayesian Inference Framework for Procedural Material Parameter Estimation.}
Procedural material models have been gaining traction in many applications thanks to their flexibility, compactness, and easy editability.
In Chapter \ref{cpt:bayesian}, we explore the inverse rendering problem of procedural material parameter estimation from photographs, presenting a unified view of the problem in a Bayesian framework. In addition to computing point estimates of the parameters by optimization, our framework uses a Markov Chain Monte Carlo approach to sample the space of plausible material parameters, providing a collection of plausible matches that a user can choose from, and efficiently handling both discrete and continuous model parameters. To demonstrate the effectiveness of our framework, we fit procedural models of a range of materials---wall plaster, leather, wood, anisotropic brushed metals and layered metallic paints---to both synthetic and real target images.

The dissertation is organized as follows. We first introduce the basic background on light transport and ********** in Chapter \ref{cpt:background}. From Chapters \ref{cpt:layeredbsdf} to \ref{cpt:bayesian}, we present technical details of our ****, ****, **** and ****, respectively. Finally, we present our conclusion and discuss future research directions in Chapter \ref{cpt:conclusion}.
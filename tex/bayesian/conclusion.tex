\section{Conclusion}
\label{sec:conclusion}
%
%We have introduced a differentiable Bayesian framework for parameter estimation of procedural material models demonstrating various phenomena and optical properties. Procedurals have been gaining significant traction in the industry, because they can cover large areas without repetition, and are easily editable.
%We introduced a differentiable forward evaluator, implemented using PyTorch. We proposed several \emph{summary functions}, enabling us to compare a synthetic simulation image to a target image (photo) effectively, by comparing their summary vectors. We have shown that a neural summary function \cite{Gatys2015,Aittala2016} works well in the procedural material setting, and generally outperforms classical summary functions.
%
Procedural material models have become increasingly more popular in the industry, thanks to their flexibility, compactness, as well as easy editability.
In this paper, we introduced a new computational framework to solve the inverse problem: the inference of procedural model parameters based on a single input image.

The first major ingredient to our solution is a \emph{Bayesian framework}, precisely defining the posterior distribution of the parameters, combining four components (priors, procedural material model, rendering operator, summary function). The second ingredient is an \emph{Bayesian inference approach} that leverages MCMC sampling to sample posterior distributions of procedural material parameters.  This technique enjoys the generality to handle both continuous and discrete model parameters and provides users additional information beyond single point estimates and allows a cleaner extension to handle discrete parameters.

In the future, we would like to increase the complexity of the models supported even further, to handle materials like woven fabrics, transmissive BTDFs, and more. Finally, extensions to our approach could be used to estimate parameters of procedural models beyond materials, including geometry and lighting, as long as the parameters could be differentiated.

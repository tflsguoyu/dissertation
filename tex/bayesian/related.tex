\section{Related Work}
\label{sec:bayesian:related}

We review previous work on material parameter estimation in computer graphics and vision, as well as on Markov-Chain Monte Carlo (MCMC) methods in Bayesian inference.

\paragraph{SVBRDF capture.} A large amount of previous work focuses on acquisition of material data from physical measurements. The methods generally observe the material sample with a fixed camera position, and solve for the parameters of a spatially-varying BRDF model such as diffuse albedo, roughness (glossiness) and surface normal. They differ in the number of light patterns required and their type; the patterns used include moving linear light \cite{gardner2003linear}, Gray code patterns \cite{francken2009gloss} and spherical harmonic illumination \cite{ghosh2009estimating}. In these approaches, the model and its optimization are specific to the light patterns and the optical setup of the method, as general non-linear optimization was historically deemed  inefficient and not robust enough.

More recently, Aittala et al. \cite{aittala2013practical} captured per-pixel SVBRDF data using Fourier patterns projected using an LCD screen; their algorithm used a fairly general, differentiable forward evaluation model, which was inverted in a maximum a-posteriori (MAP) framework. Later work by Aittala et al. \cite{aittala2015two,aittala2016reflectance} found per-pixel parameters of stationary spatially-varying SVBRDFs from two-shot and one-shot flash-lit photographs, respectively. In the latter case, the approach used a neural Gram-matrix texture descriptor based on the texture synthesis and feature transfer work of Gatys \cite{gatys2015neural,gatys2016image} to compare renderings with similar texture patterns but without pixel alignment. We demonstrate that this descriptor makes an excellent summary function within our framework; in fact, the approach works well in our case, as the procedural nature of the model serves as an additional implicit prior, compared to per-pixel approaches. On the other hand, our forward evaluation process is more complex than Aittala et al., since it also includes the procedural material generation itself.

Recent methods by Deschaintre et al. \cite{deschaintre2018single}, Li et al. \cite{li2018materials} have been able to capture non-stationary SVBRDFs from a single flash photograph by training an end-to-end deep convolutional network. Gao et al. \cite{gao2019deep} introduced an auto-encoder approach, optimizing the appearance match in the latent space. All of these approaches estimate per-pixel parameters of the microfacet model (diffuse albedo, roughness, normal), and are not obviously applicable to estimation of procedural model parameters, nor to more advanced optical models (significant anisotropy, layering or scattering).

\paragraph{Procedural material parameter estimation.} Focus on estimating the parameters of procedural models has been relatively rare. The dual-scale glossy parameter estimation work of Wang et al. \cite{wang2011estimating} finds, under step-edge lighting, the parameters of a bumpy surface model consisting of a heightfield constructed from a Gaussian noise power spectrum and global microfacet material parameters. Their results provide impressive accuracy, but the solution is highly specialized for this material model and illumination.

Recently, Hu et al. \cite{hu2019novel} introduced a method for inverse procedural material modeling that treats the material as a black box, and trains a neural network mapping images to parameter vector predictions. The training data comes from evaluating the black box model for random parameters. In our experiments, this approach was less accurate; our fully differentiable models can achieve higher accuracy fits and can be used to explore posterior distributions through sampling. In a sense, this neural prediction method could be seen as orthogonal to ours, as we could use it for initialization of our parameter vector, continuing with our MCMC sampling.

\paragraph{Optical parameters of fiber-based models.} Several approaches for rendering of fabrics model the material at the microscopic fiber level \cite{zhao2011building,zhao2016fitting,leaf2018interactive}. However, the optical properties of the fibers (e.g. roughness, scattering albedo) have to be chosen separately to match real examples. Zhao et al. \cite{zhao2011building} use a simple but effective trick of matching the mean and standard deviation (in RGB) of the pixels in a well-chosen area of the target and simulated image. Khungurn et al. \cite{khungurn2015matching} have extended this approach with a differentiable volumetric renderer, combined with a stochastic gradient descent; however, their method is still specific to fiber-level modeling of cloth.

\paragraph{Bayesian inference and MCMC.} A variety of methods used across the sciences are Bayesian in nature; in this chapter, we specifically explore Bayesian inference for parameter estimation through Markov-Chain Monte Carlo (MCMC) sampling of the posterior distribution. Provided a nonnegative function~$f$, MCMC techniques can draw samples from the probability density proportional to the given function~$f$ without knowing the normalization factor. Metropolis-Hastings \cite{hastings1970monte} is one of the most widely used MCMC sampling methods. If $f$ is differentiable, the presence of gradient information leads to more efficient sampling methods such as Hamiltonian Monte Carlo (HMC) \cite{neal2011mcmc,betancourt2017conceptual} and Metropolis-adjusted Langevin algorithm (MALA) \cite{roberts1996exponential}.
Our inference framework is not limited to any specific MCMC sampling technique.
In practice, our implementation handles discrete model parameters using MH and continuous ones using MALA (with preconditioning \cite{chen2016bridging}). We opt MALA for its simpler hyper-parameter tweaking (compared to HMC).

\paragraph{MCMC applications in graphics and vision.} Markov chain Monte Carlo techniques have been heavily studied in rendering, though not for Bayesian inference, but rather for sampling light transport paths with probability proportional to their importance; notably Metropolis light transport \cite{veach1997metropolis} and its primary sample space variant \cite{kelemen2002simple}. Much further work has built on these techniques, including more recent work that uses a variant of Hamiltonian Monte Carlo \cite{li2015anisotropic}. However, all of these approaches focus on better sampling for traditional rendering, rather than parameter estimation in inverse rendering tasks.

In computer vision, Bayesian inference with MCMC has been used for the inverse problems of scene understanding. A notable previous work is Picture \cite{kulkarni2015picture}, a probabilistic system and programming language for scene understanding tasks, for example (though not limited to) human face and body pose estimation. The programming language is essentially used to specify a forward model (e.g., render a face in a given pose), and the system then handles the MCMC sampling of the posterior distribution through a combination of sampling (proposal) techniques. This is closely related to the overall design of our system. However, the Picture system does not appear to be publicly available, and our application is fairly distant from its original goals.


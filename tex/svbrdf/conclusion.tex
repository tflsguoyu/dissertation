\section{Conclusion}
\label{sec:svbrdf:conclusion}

We propose a novel method for acquiring SVBRDFs from a small number of input images, typically 3 to 7, captured using a hand-held mobile phone.
We use an optimization framework that leverages a powerful material prior, based on a generative network, MaterialGAN, trained to synthesize plausible SVBRDFs.
MaterialGAN learns correlations in SVBRDF parameters and provides local and global regularization to our optimization.
This produces high-quality SVBRDFs that accurately reconstruct the input images, and because of our MaterialGAN prior, lie on a plausible material manifold.
As a result, our reconstructions generalize better to novel views and lighting than previous state-of-the-art methods.

We believe that our work is only a first step toward GAN-based material analysis and synthesis and our experiments suggest many avenues for further exploration including improving material latent spaces and optimization techniques using novel architectures and losses, learning disentangled and editable latent spaces, and expanding beyond our current isotropic BRDF model.



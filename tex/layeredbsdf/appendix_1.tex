\section{Detailed Derivations}
\label{sec:layeredbsdf:derivation}

We now provide detailed derivations for the key equations in \S\ref{sec:layeredbsdf:pathformulation}.

\paragraph{Position-free radiative transfer equation.}
Traditionally, the radiative transfer equation (RTE) involves an integral over free-flight distance $t$:
\begin{equation}
\label{eq:IRTE0}
  L_v(z, \bmomega) = S(z, \bmomega) + \int_0^{t'} \exp(-t \sigma_t) \, \int_{\bbSS} \hat f_p(\bmomega', \bmomega) \, L_v(z', \bmomega') \,\intd \bmomega' \intd t,
\end{equation}
where $z' := z - t\cos\bmomega$ and $t'$ denotes the distance between $z$ and the closest layer boundary.
Since $t = (z - z')/\cos\bmomega$, changing the integration variable from $t$ to $z'$ in Eq. \eqref{eq:IRTE0} yields an additional factor of $(\cos\bmomega)^{-1}$ which in turn gives our position-free RTE \eqref{eq:IRTE}.
Notice that the change-of-variable ratio only appears within the integration (and not in the source term $S$).

\paragraph{Cosines in path contribution.}
The contribution $f$ of a light path $\bar{x}$ can be obtained by repeatedly expanding the rendering equation \eqref{eq:RE} and our position-free RTE \eqref{eq:IRTE}.

Similar to the traditional path integral formulation, for each vertex $z_i$ corresponding to an interface event (i.e., reflection or refraction), a cosine term $|\cos\bmd_i|$ is needed to ensure the measure of projected solid angle.

On the other hand, a segment of our light path connecting two depths $z_i$ and $z_{i + 1}$ via direction $\bmd_i$ can yield an additional $|\cos\bmd_i|^{-1}$ when $z_{i + 1}$ corresponds to a volumetric scattering.
Thus, for each $i$, the path contribution involve a factor of $|\cos\bmd_i|^{\alpha_i}$ with:
\begin{itemize}
	\item $\alpha_i = 1$ if $z_i$ and $z_{i + 1}$ are both on interfaces;
	\item $\alpha_i = 0$ (i.e., no $\cos\bmd_i$ term) if (i)~$z_i$ is volumetric and $z_{i + 1}$ lies on an interface (so that no $\cos\bmd_i$ terms appear during expansion), or (ii)~$z_i$ is interfacial and $z_{i + 1}$ is volumetric (so that both $|\cos\bmd_i|$ and $|\cos\bmd_i|^{-1}$ are present, canceling out each other);
	\item $\alpha_i = -1$ if $z_i$ and $z_{i + 1}$ are both volumetric vertices.
\end{itemize}
Eq. \eqref{eqn:seg_contrib_cosine} provides a compact way to encode these rules. 

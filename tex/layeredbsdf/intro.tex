\section{Introduction}
\label{sec:layeredbsdf:intro}

Physically-based shading models have become mature and commonplace in recent years across a number of rendering applications, within entertainment, architecture, and industrial design. 
However, we are seeing constant progress in the area of material reflection and scattering models, aiming to achieve higher physical realism and to enable more effective material content creation.

Many real world materials are comprised of thin layers with varying compositions. For example, metallic paint is a dielectric coating covering a metallic substrate composed of randomly oriented aluminum flakes; the absorption and scattering properties of the dielectric layer give the material its color and modify its directional scattering properties as well.
Many biological materials (e.g. plant leaves) are also layered, and their appearance is a complex combination of the absorption properties, scattering phase function, air-material interface roughness, and thickness variation.
Different characteristics of such interfaces and volumetric scattering properties can produce richly diverse material appearances from anisotropic highlights to complex textures. Furthermore, detailed layer thickness variations, scratches and bumps on the layer interfaces give these materials additional richness. Accurately understanding and simulating these interactions is therefore key to further progress in the rendering of materials.

However, explicitly simulating light-layer interactions by modeling the full geometry of these layers would be very expensive and cumbersome. 
The complex and spatially varying interface and internal microgeometries are much too costly to describe and simulate using standard 3D scene modeling tools such as triangle meshes and volumetric grids.
Furthermore, due to the presence of multiple refractive interfaces, it can be very challenging to correctly construct light transport paths that connect light scattering locations to light sources, a key operation in most practical Monte Carlo rendering systems. Cheap approximations to these light transport problems (e.g. ignoring refraction, or composing layers using simple blending) are not sufficient to achieve true realism.

A few techniques have been developed to address this problem. Weidlich and Wilkie \cite{weidlich2007arbitrarily} construct a simple and flexible analytical model. However, significant approximations are necessary; interface roughness is not fully handled for transmission, and no volumetric scattering is supported. The work of Belcour \cite{belcour2018efficient} recently introduced a more advanced approach based on tracking low-order moments of the BSDF lobes; however, it still introduces some approximations and limitations. On the other hand, Jakob et al. \cite{jakob2014comprehensive} (with a recent follow-up \cite{zeltner2018layer}) introduce a solution that is very accurate, but expensive: it represents BSDFs as discretized datasets and relies on expensive Fourier-domain operations on these to implement layer composition and thickness adjustment. This makes free spatial variation of the layer properties prohibitively expensive: a significant limitation in practice.

In this chapter, we introduce a new layered BSDF model without the above limitations. Our model provides an accurate, unbiased solution; to our knowledge, it is the only such model.
Unlike previous work, we do not attempt to derive an analytic model for the BSDF lobe shapes. Instead, inside the evaluation and sampling routines of the layered BSDF, we run a Monte Carlo simulation of light transport within flat slabs.
This is substantially faster than explicitly constructing the layer geometry, because no expensive scene ray tracing is required.
Our model computes an accurate solution of the layered light transport problem.
It is based on physical interface and volume scattering models, conserves energy and is reciprocal when possible. It can also be easily integrated into standard Monte Carlo rendering systems.
This requires no precomputation and thus can efficiently handle spatially varying appearances. It also supports the full range of editability of the layer properties, both interface and volumetric, and allows anisotropy in both interface BSDFs and phase functions. In fact, the only limiting assumption of our model is the layer assumption itself.

Our solution is fundamentally more powerful at constructing light transport paths than generic transport algorithms (e.g standard path tracing, bidirectional or Metropolis transport); see Figure \ref{fig:layeredbsdf:equal_time_compare}. 
\begin{figure}[h]
	\centering
	\setlength{\resLen}{1.04in}
	\addtolength{\tabcolsep}{-4.5pt}
	\begin{tabular}{cccccc}
		\begin{overpic}[width=\resLen]{layeredbsdf/validations/compare1/plane_fullsim_ref_3_2h.jpg} 
		\end{overpic} &
		\begin{overpic}[width=\resLen]{layeredbsdf/validations/compare1/plane_10s_pt_98spp.jpg} 
			\put(2,3){\bfseries \color{white} 98 spp} 
		\end{overpic} &
		\begin{overpic}[width=\resLen]{layeredbsdf/validations/compare1/plane_10s_bdpt_35spp.jpg} 
			\put(2,3){\bfseries \color{white} 35 spp} 
		\end{overpic} &
		\begin{overpic}[width=\resLen]{layeredbsdf/validations/compare1/plane_10s_mlt_280spp.jpg}
			\put(2,3){\bfseries \color{white} 280 spp} 
		\end{overpic} &
		\begin{overpic}[width=\resLen]{layeredbsdf/validations/compare1/plane_10s_uni_56spp.jpg} 
			\put(2,3){\bfseries \color{white} 56 spp} 
		\end{overpic} &
		\begin{overpic}[width=\resLen]{layeredbsdf/validations/compare1/plane_10s_bi_26spp.jpg} 
			\put(2,3){\bfseries \color{white} 26 spp} 
		\end{overpic}
		\\
		\begin{overpic}[width=\resLen]{layeredbsdf/validations/compare1/plane_fullsim_ref_4_8h.jpg} 
		\end{overpic} &
		\begin{overpic}[width=\resLen]{layeredbsdf/validations/compare1/plane_10s_pt_60spp.jpg} 
			\put(2,3){\bfseries \color{white} 60 spp} 
		\end{overpic} &
		\begin{overpic}[width=\resLen]{layeredbsdf/validations/compare1/plane_10s_bdpt_25spp.jpg} 
			\put(2,3){\bfseries \color{white} 25 spp} 
		\end{overpic} &
		\begin{overpic}[width=\resLen]{layeredbsdf/validations/compare1/plane_10s_mlt_80spp.jpg} 
			\put(2,3){\bfseries \color{white} 80 spp} 
		\end{overpic} &
		\begin{overpic}[width=\resLen]{layeredbsdf/validations/compare1/plane_10s_uni_15spp.jpg} 
			\put(2,3){\bfseries \color{white} 15 spp} 
		\end{overpic} &
		\begin{overpic}[width=\resLen]{layeredbsdf/validations/compare1/plane_10s_bi_19spp.jpg} 
			\put(2,3){\bfseries \color{white} 19 spp} 
		\end{overpic}
		\\
		\small Reference &
		\small Standard PT &
		\small Standard \footnotesize BDPT &
		\small Standard MLT &
		\small Our unidir. &
		\small Our bidir.
	\end{tabular}
	\caption[Equal-time comparisons]{\label{fig:layeredbsdf:equal_time_compare}
		\textbf{Equal-time comparisons} of our unidirectional and bidirectional approach to standard transport algorithms, on a simple flat layered configuration lit by a small area light. 
		For standard PT, BDPT and MLT, results are all generated using 3D tracing by applying these algorithms in a simple 3D scene containing a very large slab with flat interfaces.
		\textbf{Top}: A single slab with Henyey-Greenstein scattering between two interfaces, where our estimators perform similarly, but both significantly outperform path tracing, bidirectional and Metropolis transport. \textbf{Bottom}: A more complex configuration with two slabs and three interfaces; both media are using an anisotropic microflake phase function \cite{jakob2010radiative}. Our bidirectional estimator is a clear winner in this case.
		The references are generated using standard PT with 100K spp, and all the other images are rendered in 10 seconds.
	}
\end{figure}


We introduce a modified path integral framework for light transport in flat slabs, superior to the standard path formulation in this setting. Because it is based on a product of solid angle instead of area measures, it does not contain the high-variance geometry terms needed in standard algorithms. We introduce two simulation techniques within this formulation: the first is analogous to a forward path tracer with next event estimation through layer boundaries and multiple importance sampling; the second is a fully bidirectional estimator. We show the capabilities of this solution on a number of examples, featuring multiple layers with surface and volumetric scattering. Our examples show spatial variation in all parameters: surface BSDF, volume and phase function parameters, layer thickness and surface normal. See Figure \ref{fig:layeredbsdf:teaser}.


\begin{figure}[!ht]
	\centering
	\setlength{\resLen}{2.1in}
	\addtolength{\tabcolsep}{-3.5pt}
	\begin{tabular}{ccc}
		\begin{overpic}[width=\resLen]{layeredbsdf/results/satin_spp128.jpg}
			\put(2,60){\color{white} \textbf{(a)}}
		\end{overpic}
		&
		\begin{overpic}[width=\resLen]{layeredbsdf/results/twill_128spp.jpg}
			\put(2,60){\color{white} \textbf{(b)}}
		\end{overpic}
		&
		\begin{overpic}[width=\resLen]{layeredbsdf/results/velvet_spp128.jpg}
			\put(2,60){\color{white} \textbf{(c)}}
		\end{overpic}
	\end{tabular}
	\caption[Anisotropic media within layers]{\label{fig:layeredbsdf:cloth}
		\textbf{Anisotropic media within layers.}
		Our layered BSDF offers the generality to use anisotropic layer media with microflake phase functions.
		This example shows three fabrics modeled with our BSDF model with anisotropic layer media: \textbf{(a)} satin; \textbf{(b)} twill; and \textbf{(c)} velvet.
	}
\end{figure}    
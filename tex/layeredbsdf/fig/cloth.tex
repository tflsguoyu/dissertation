\begin{figure}[h]
	\centering
	\setlength{\imgWidth}{2.1in}
	\addtolength{\tabcolsep}{-3.5pt}
	\begin{tabular}{ccc}
		\begin{overpic}[width=\imgWidth]{layeredbsdf/results/satin_spp128.jpg}
			\put(2,60){\bfseries \color{white} (a) Satin}
		\end{overpic}
		&
		\begin{overpic}[width=\imgWidth]{layeredbsdf/results/twill_128spp.jpg}
			\put(2,60){\bfseries \color{white} (b) Twill}
		\end{overpic}
		&
		\begin{overpic}[width=\imgWidth]{layeredbsdf/results/velvet_spp128.jpg}
			\put(2,60){\bfseries \color{white} (c) Velvet}
		\end{overpic}
	\end{tabular}
	\caption[Anisotropic media within layers]{\label{fig:layeredbsdf:cloth}
		\textbf{Anisotropic media within layers.}
		Our layered BSDF offers the generality to use anisotropic layer media with microflake phase functions.
		This example shows three fabrics modeled with our BSDF model with anisotropic layer media: (a) satin; (b) twill; and (c) velvet.
	}
\end{figure}    
\thesistitle{Microscale-based Macro-appearance Rendering and Its Inverse Problem}

%"Dissertation" for PhD, "Thesis" for master's
\documenttitle{Dissertation}

\degreename{Doctor of Philosophy}

% Use the wording given in the official list of degrees awarded by UCI:
% http://www.rgs.uci.edu/grad/academic/degrees_offered.htm
\degreefield{Computer Science}

% Your name as it appears on official UCI records.
\authorname{Yu Guo}

% Use the full name of each committee member and full title 
% (e.g. Professor/Associate Professor).
\committeechair{Professor Shuang Zhao}
\othercommitteemembers
{
  Professor Gopi Meenakshisundaram\\
  Professor Charless Fowlkes
}

\degreeyear{2021}

\copyrightdeclaration
{
  {\copyright} {\Degreeyear} \Authorname
}

% If you have previously published parts of your manuscript, you must list the
% copyright holders; see Section 3.2 of the UCI Thesis and Dissertation Manual.
% Otherwise, this section may be omitted.
% \prepublishedcopyrightdeclaration
% {
% 	Chapter 4 {\copyright} 2003 Springer-Verlag \\
% 	Portion of Chapter 5 {\copyright} 1999 John Wiley \& Sons, Inc. \\
% 	All other materials {\copyright} {\Degreeyear} \Authorname
% }

% The dedication page is optional
% (comment out to exclude).
\dedications
{
  % (Optional dedication page)
  
  To Myself and My Family
}

\acknowledgments
{
  I would like to thank my advisor Shuang Zhao for his patient guidance and unconditional support.
  
  I am sincerely grateful to Milo\v{s} Ha\v{s}an for his inspiration and offering me opportunities for my internships in Autodesk and Adobe. I would also like to thank my other committee members Gopi Meenakshisundaram and Charless Fowlkes for the constructive advice they have provided. 
  
  I want to thank all the collaborator from my publications.
  And my intership mentors.
  
  I would like to thank my labmates and other friends.
  
  Finally, I thank my wife and my parents.
  
  This work was supported in part by NSF grant IIS-1813553, Autodesk Inc., Adobe Inc., University of Zaragoza (Spain) and Department of Computer Science, UC Irvine.
  
  Chapter \ref{cpt:layeredbsdf} is based on the material as it appears in ACM Transactions on Graphics, 2018
  (“Position-Free Monte Carlo Simulation for Arbitrary Layered BSDFs”, Yu Guo, Milo\v{s} Ha\v{s}an and Shuang Zhao). The dissertation author was the primary investigator and author of this paper.
  
  Chapter \ref{cpt:waveoptics} is based on the material as it appears in ACM Transactions on Graphics, 2021
  ("Beyond Mie Theory: Systematic Computation of Bulk Scattering Parameters based on Microphysical Wave Optics", Yu Guo, Adrian Jarabo and Shuang Zhao). The dissertation author was the primary investigator and author of this paper.
  
  Chapter \ref{cpt:svbrdf} is based on the material as it appears in ACM Transactions on Graphics, 2020
  (“MaterialGAN: Reflectance Capture using a Generative SVBRDF Model”, Yu Guo, Cameron Smith, Milo\v{s} Ha\v{s}an, Kalyan Sunkavalli and Shuang Zhao). The dissertation author was the primary investigator and author of this paper.
  
  Chapter \ref{cpt:bayesian} is based on the material as it appears in Computer Graphics Forum, 2020
  (“A Bayesian Inference Framework for Procedural Material Parameter Estimation”, Yu Guo, Milo\v{s} Ha\v{s}an, Lingqi Yan and Shuang Zhao). The dissertation author was the primary investigator and author of this paper.

  This dissertation is based on a \LaTeX~template for thesis and dissertation documents at UC Irvine \cite{uci-thesis-latex}.
}


% Some custom commands for your list of publications and software.
\newcommand{\mypubentry}[3]{
  \begin{tabular*}{1\textwidth}{@{\extracolsep{\fill}}p{4.5in}r}
    \textbf{#1} & \textbf{#2} \\ 
    \multicolumn{2}{@{\extracolsep{\fill}}p{.95\textwidth}}{#3}\vspace{6pt} \\
  \end{tabular*}
}
\newcommand{\mysoftentry}[3]{
  \begin{tabular*}{1\textwidth}{@{\extracolsep{\fill}}lr}
    \textbf{#1} & \url{#2} \\
    \multicolumn{2}{@{\extracolsep{\fill}}p{.95\textwidth}}
    {\emph{#3}}\vspace{-6pt} \\
  \end{tabular*}
}

% Include, at minimum, a listing of your degrees and educational
% achievements with dates and the school where the degrees were
% earned. This should include the degree currently being
% attained. Other than that it's mostly up to you what to include here
% and how to format it, below is just an example.
%
% CV is required for PhD theses, but not Master's
% comment out to exclude
\curriculumvitae
{

\textbf{EDUCATION}
  
  \begin{tabular*}{1\textwidth}{@{\extracolsep{\fill}}lr}
    \textbf{Doctor of Philosophy in Computer Science} & \textbf{2016 -- 2021} \\
    \vspace{6pt}
    University of California, Irvine & \emph{Irvine, CA, US} \\
    \textbf{Master of Science in Computational Sciences} & \textbf{2010 -- 2013} \\
    \vspace{6pt}
    University of Chinese Academy of Sciences & \emph{Beijing \& Shenzhen, China} \\
    \textbf{Bachalar of Science in Applied Mathematics} & \textbf{2006 -- 2010} \\
	\vspace{6pt}
	Central South University & \emph{Changsha, China} \\
  \end{tabular*}

\vspace{12pt}
\textbf{RESEARCH EXPERIENCE}

  \begin{tabular*}{1\textwidth}{@{\extracolsep{\fill}}lr}
    \textbf{Research Associate} & \textbf{2013 -- 2016} \\
    \vspace{6pt}
    Nanyang Technological University & \emph{Singapore} \\
  \end{tabular*}

\pagebreak

\textbf{REFEREED PUBLICATIONS}

  \mypubentry{Position-Free Monte Carlo Simulation for Arbitrary Layered BSDFs}{2018}{ACM Transactions on Graphics}

  \mypubentry{MaterialGAN: Reflectance Capture using a Generative SVBRDF Model}{2020}{ACM Transactions on Graphics}

  \mypubentry{A Bayesian Inference Framework for Procedural Material Parameter Estimation}{2020}{Computer Graphics Forum}

  \mypubentry{Beyond Mie Theory: Systematic Computation of Bulk Scattering Parameters based on Microphysical Wave Optics}{2021}{ACM Transactions on Graphics}

}

% The abstract was previously limited to a maximum of 350 words, 
% but the UCI manual at https://etd.lib.uci.edu/electronic/td2e#2.2.1.
% currently does not indicate that there is any word limit for the abstract
\thesisabstract
{
  Physically-based rendering has become mature and commonplace in recent decades. However, the rendered results look artificial and overly perfect. Better realism needs higher fidelity detailed geometry or model complexity, which substantially increases computational power and human works. To achieve higher physical realism and enable more effective material content creation, many techniques are developed in material reflection and scattering models. We put emphasis on accurately representing and reproducing the rich visual world from \emph{micro}-level details to overall (\emph{macro}) appearance.

  The first half of the dissertation focuses on building the bridge from \emph{micro} to \emph{macro} world: we present an accurate appearance model for layered materials derived from microstructures to define their optical behavior and a general framework of bulk scattering in participating medium which considers the microscale effects. Contradictory, in the second half, we discuss the inverse problem of retrieving the micro parameters from photo-captured materials. 

  Our first work introduces a new unbiased layered BSDF model based on Monte Carlo simulation, whose only assumption is the layer assumption itself. Our novel position-free path formulation is fundamentally more powerful at constructing light transport paths than generic light transport algorithms applied to the particular case of flat layers. We introduce two techniques for sampling the position-free path integral, a forward path tracer with next-event estimation and a full bidirectional estimator. We show several examples featuring multiple layers with surface and volumetric scattering, surface and phase function anisotropy, and spatial variation in all parameters.

  Our second work presents a generalized framework capable of systematically and rigorously computing bulk scattering parameters beyond the far-field assumption of the Lorenz-Mie theory. Our technique accounts for microscale wave-optics effects such as diffraction and interference and interactions between nearby particles. Our framework is general, can be plugged in any renderer supporting Lorenz-Mie scattering, and allows arbitrary packing rates and particle correlation; we demonstrate this generality by computing bulk scattering parameters for many materials, including anisotropic materials and correlated media.

  Finally, we present \emph{MaterialGAN}, a deep generative convolutional network based on StyleGAN2, trained to synthesize realistic SVBRDF parameter maps. We show that MaterialGAN can be used as a robust material prior in an inverse rendering framework: we optimize its latent representation to generate material maps that match the appearance of the captured images when rendered. 

  Furthermore, we explore the inverse rendering problem of procedural material parameter estimation from photographs, presenting a unified view of the problem in a Bayesian framework. In addition to computing point estimates of the parameters by optimization, our framework uses a Markov Chain Monte Carlo approach to sample the space of plausible material parameters.
}


%%% Local Variables: ***
%%% mode: latex ***
%%% TeX-master: "thesis.tex" ***
%%% End: ***
